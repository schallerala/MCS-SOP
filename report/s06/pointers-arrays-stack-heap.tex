\documentclass[12pt]{article}

\usepackage[
    a4paper,
    hmargin=2cm, vmargin=1.3cm,
    headheight=56.6pt, % as per the warning by fancyhdr
    headsep=10pt,
    includehead, includefoot
]{geometry}

\usepackage[english]{babel}
\usepackage[T1]{fontenc}
\usepackage[utf8x]{inputenc}

\usepackage{fancyhdr} % extensice control of page headers and footers
\usepackage{graphicx}
\usepackage{parskip} % add space between paragraphs and remove first line's indent

\usepackage[dvipsnames,table]{xcolor} % color the text for the color code: https://en.wikibooks.org/wiki/LaTeX/Colors
\usepackage[bottom]{footmisc} % to place footnotes to the bottom of the page
% \usepackage{pdfpages} % \includepdf[pages=-]{myfile.pdf}: all pages, or `pages={1}`
\usepackage{amsmath} % provides math environments
\usepackage{amssymb} % add math symbols
\usepackage{cancel} % https://tex.stackexchange.com/a/75530/164820
\usepackage{ifthen} % provides \ifthenelse test
\usepackage{xifthen} % provides \isempty test
\usepackage{float} % improve the figure placement in the document
\usepackage[normalem]{ulem} % used to strike through: https://jansoehlke.com/2010/06/strikethrough-in-latex/
\usepackage{multirow}
\usepackage{hhline} % https://tex.stackexchange.com/a/39766/164820
\usepackage[acronym,nonumberlist,nogroupskip]{glossaries} % to handle acronyms

% Fix non breakable links: https://tex.stackexchange.com/a/3034/164820
\PassOptionsToPackage{hyphens}{url}\usepackage{hyperref}
\usepackage[inline]{enumitem} % allow to set some margin to list env's
\usepackage{tcolorbox}

\usepackage{listings}
\usepackage{tikz}
\usetikzlibrary{calc}
\usepackage{pgfplots}


% ---------------- SETUP HEADER & FOOTER ----------------------
\pagestyle{fancy}
% \fancyhf{} % sets both header and footer to nothing
% \renewcommand{\headrulewidth}{0pt} % remove headers bottom line


% Define variables
\makeatletter
\newcommand{\prof}[1]{\def\@prof{#1}}
\newcommand{\theprof}{\@ifundefined{@prof}{}{\\Prof.: \@prof}}
\newcommand{\student}[1]{\def\@student{#1}}
\newcommand{\thestudent}{\@ifundefined{@student}{}{\@student}}
\newcommand{\seriesnumber}[1]{\def\@seriesnumber{#1}}
\newcommand{\theseriesnumber}{\@ifundefined{@seriesnumber}{}{Series \@seriesnumber}}
\makeatother


% https://tex.stackexchange.com/a/218450/164820
\lstnewenvironment{algorithm}[1][] %defines the algorithm listing environment
{
    % \refstepcounter{nalg} %increments algorithm number
    % \captionsetup{labelformat=algocaption,labelsep=colon} %defines the caption setup for: it ises label format as the declared caption label above and makes label and caption text to be separated by a ':'
    \lstset{ %this is the stype
        mathescape=true,
        frame=tB,
        numberstyle=\tiny,
        basicstyle=\scriptsize,
        keywordstyle=\color{black}\bfseries\em,
        keywords={,input, output, return, datatype, initialize, do, function, in, if, else, then, foreach, break, while, begin, end} %add the keywords you want, or load a language as Rubens explains in his comment above.
        % numbers=left,
        xleftmargin=.04\textwidth,
        #1 % this is to add specific settings to an usage of this environment (for instnce, the caption and referable label)
    }
}
{}



\newcommand{\unilogo}[1][0.16\textwidth]{\includegraphics[width=#1]{logo_uni}}

\newcommand*{\fullref}[1]{\hyperref[{#1}]{\autoref*{#1}} \textit{\nameref*{#1}}} % https://tex.stackexchange.com/a/121871
\newcommand*{\simpleref}[1]{\hyperref[{#1}]{{\autoref*{#1}}}}
\newcommand*{\namedref}[2]{\hyperref[{#1}]{\textit{#2}}}
\newcommand*{\visitedurl}[2]{\url{#1}\hspace{0.6em}(visited on #2)}
\newcommand*{\footnotelink}[2]{\footnote{\visitedurl{#1}{#2}}}
\newcommand*{\footnotetextlink}[2]{\footnotetext{\visitedurl{#1}{#2}}}

\newcommand*{\notice}[2][Notice]{
Notice:
\begin{itemize}
    \vspace{-0.2cm}
    \item #2
\end{itemize}
}


\newcommand{\yellowInBlue}[1]{{\colorbox{blue!50}{\color{yellow}\textbf{#1}}}}

\newcommand*{\todo}[1]{
    \ifthenelse{\isempty{#1}}
        {\yellowInBlue{TODO}}
        {\yellowInBlue{TODO #1}} % else
}



\newcounter{question}[section]
\newcommand{\questioncolor}{\color{CadetBlue}}
\def\newquestion#1{
    \refstepcounter{question}
    {
        \textbf{Question~\arabic{question}.}
        \questioncolor
        \emph{#1}
    }
}
\def\question#1{
    {
        \questioncolor
        \emph{#1}
    }
}


\setlist[description]{leftmargin=0.5cm,labelindent=0.5cm,parsep=0.4ex}

% Table layout
\def\arraystretch{1.5} % height of each row is set to 1.5 relative to its default height.
\setlength{\tabcolsep}{8pt} % space between the text and the left/right border

\glsenablehyper
\hypersetup{
	colorlinks = true,
	% linkcolor = red,
	% menucolor = red,
	% filecolor = blue,
	% anchorcolor = green,
	% urlcolor = blue,
	allcolors = MidnightBlue,
	linkbordercolor = white
}

\definecolor{commentcolor}{HTML}{969896}
\definecolor{mGray}{rgb}{0.5,0.5,0.5}
\definecolor{stringcolor}{HTML}{769107}
\lstset{
    % language=csh,
    % numbers=left,
    % numbersep=5pt,
    % numberstyle=\tiny,
    aboveskip=12pt,
    basicstyle=\linespread{1.05}\small\ttfamily,
    belowskip=9pt,
    breaklines=true,
    captionpos=b,
    commentstyle=\color{commentcolor},
    extendedchars=true,
    frame=tb,
    framexbottommargin=4pt,
    framexleftmargin=17pt,
    framexrightmargin=5pt,
    keywordstyle=\color{magenta},
    numbers=none,  % still add them: numbers=left
    numberstyle=\tiny\color{mGray},
    showspaces=false,
    showstringspaces=false,
    showtabs=false,
    stringstyle=\color{stringcolor}\ttfamily,
    tabsize=2,
    xleftmargin=17pt,
}


% https://tex.stackexchange.com/a/172480/164820
% new tcolorbox environment
% #1: tcolorbox options
% #2: color
\newtcolorbox{mybox}[2][]
{
  colframe = #2!25,
  colback  = #2!10,
  coltitle = #2!20!black,
  #1,
}


\loadglsentries{../glossary}
\makeglossaries


\graphicspath{{../assets/}}


\title{
    %\vspace{-2cm}
    \unilogo[0.3\textwidth]\\[0.9cm]
    S06: Pointers and Arrays + Stack and Heap Segments, Pointers and typedef
}
\student{
    A. Schaller\\
    \texttt{16-896-375}
}
\prof{Dr. Luggen Michael}

\seriesnumber{6}
\author{\thestudent\theprof}
\lhead{\unilogo}
\rhead{\thestudent\\\theseriesnumber}

% \usepackage[backend=bibtex, style=numeric, citestyle=numeric, sorting=none, backref, backrefstyle=none]{biblatex}
% use "biber report" command to generate

% \addbibresource{biblio.bib}


% \usepgfplotslibrary{fillbetween}
\pgfplotsset{compat=newest}
% \usepackage{fontawesome}

\lstset{
    language=C,
    basicstyle=\footnotesize,
    stepnumber=3,
    firstnumber=1,
}

\raggedbottom % To avoid stretching the items to fill a page
% ---------------------- DOCUMENT ----------------------------
\begin{document}

%
%
% ---------------------------------
\section{Pointers and Arrays + Stack and Heap Segments}

\newquestion{Draw the stack and heap segments just when the PC register points to the last semicolon ; of the following compound statements (assuming that local arrays are stored in the heap segment):}

\begin{enumerate}
    \item \
    \begin{lstlisting}
int i;
int j = 1;
int *p = &j
int**q = &p
int ***r = &q
i = ***r + 1;
    \end{lstlisting}

    \item \
    \begin{lstlisting}
int i;
int tab[3];
int *p = tab;
++p;
++p;
i = p - tab;
tab[0] = 1;
(tab+1)[0] = 2;
*p = 3;
    \end{lstlisting}

    \item \
    \begin{lstlisting}
int *p = malloc(2*sizeof(int));
p[0]=4;
p[2]=5;
    \end{lstlisting}
\end{enumerate}



%
%
% ---------------------------------
\section{Complicated Declarations}

Explain in your own words:

\begin{enumerate}
    \item Declare and allocate a 2-dimensional array with 3 rows of 2 columns. In the end, will allocate $3*2 = 6$ spaces for \emph{int}s.
    \begin{lstlisting}
int a[3][2];
    \end{lstlisting}

    \item Following the `\emph{C programming Spiral Rule}'\footnotelink{http://c-faq.com/decl/spiral.anderson.html}{March 2021}(alternatively to the hint\footnotelink{https://cdecl.org}{March 2021}), $b$ is an array of size 3 of pointers to \emph{int}s. Meaning, when unreferencing the pointer, we will have access to an array holding a 3 integers.
    \begin{lstlisting}
int *b[3];
    \end{lstlisting}

    \item Still following the \emph{Spiral Rule}\footnotemark[1], the variable $c$ is a pointer to a function without any parameter returning an integer.
    \begin{lstlisting}
int (*c)();
    \end{lstlisting}

    \item $d$ holds a function which accept as parameter a pointer to a function ($e$) which doesn't accept any parameter and the pointed function $e$ returns an integer, which finally $d$ return an integer.
    \begin{lstlisting}
int d(int (*e)());
    \end{lstlisting}

    \item $f$ holds a function which doesn't accept any parameter and return a pointer to a function which also doesn't accept any parameter and return an int.
    \begin{lstlisting}
int (*f())();
    \end{lstlisting}
\end{enumerate}

%
%
% ---------------------------------
\section{typedef Definitions}

\newquestion{The following}\verb!typedef!\question{define some very common new types. Indicate their names and their corresponding defined types.}

\begin{enumerate}
    \item The following \verb!typedef! named \verb!stackt! is a synonym to pointer of type \verb!void!.
    \begin{lstlisting}
typedef void *stackt;
    \end{lstlisting}

    \item \verb!fctInt_t! is a synonym for a pointer to a function which accept a parameter integer and return an integer.
    \begin{lstlisting}
typedef int (*fctInt_t)(int);
    \end{lstlisting}

    \item \verb!fct_gen! is a synonym for a pointer to a function which accept a pointer \verb!void! as parameter and return a pointer \verb!void!.
    \begin{lstlisting}
typedef void *(*fct_gen)(void *);
    \end{lstlisting}

    \item \verb!signal! is a synonym for a function which accept 2 parameters: $1.$ an integer, $2.$ a pointer to a function which accepted an integer and return \verb!void!, and return a pointer to function accepting a integer and return \verb!void!.
    \begin{lstlisting}
typedef void ( *signal(int, void (*)(int)) )(int);
// typedef void (* f_1)(int);
// typedef f_1 signal(int, void (*)(int));
    \end{lstlisting}
\end{enumerate}


%
%
% ---------------------------------
\section{Pointer to Function + typedef}

\question{Consider the following program}

\begin{lstlisting}
typedef int (*mathFunc_t)(int, int); // definition of type mathFunc_t

int add(int a, int b) {
    return a + b;
}

int mult(int a, int b) {
    return a * b;
}

int compute(mathFunc_t f, int a, int b) {
    return f(a, b);
}

int main() {
    mult(add(2, 4), 8);
    compute(mult, compute(add, 2, 4), 8);
    return 0;
}
\end{lstlisting}


\begin{enumerate}
    \item \newquestion{What is the return value of mult() and compute()?}

    It prints twice the value $\boldsymbol{48}$.

    \item \newquestion{Draw the simplified stack segments when the PC register points respectively to the last semicolon ; of the function calls in line 1 and 2 of main().}


\end{enumerate}


\end{document}